\pagebreak
    \hspace{0pt}
    \vfill
    \addcontentsline{toc}{section}{Abstract}
    \section*{Abstract}
    
    The polaron is a quasiparticle made of an electron and a cloud of phonons coupled to it through the electron-phonon interaction. This quasiparticle state 
    is common at the bottom of the conduction band or at the top of the valence band (hole polaron) in ionic semiconductors and insulators such as III-V compounds (GaAs, GaP, AlAs) and 
    oxides (BaO, CaO). In this thesis we will be focusing on the large polaron model, also known as the Froehlich polaron, which assumes that the material can be modelled as 
    a dielectric continuum. The main features of the Froehlich model will be discussed, including also a generalization for our more complex case, and then the Many Body formalism will be 
    explained together with the main aspects of the Monte Carlo method (introducing also Markov chains), in order to provide the theoretical basis for the Diagrammatic Monte Carlo method applied to the large polaron.\\
    The simulations performed using Diagrammatic Monte Carlo were used to compute the ground state energy and polaron effective masses in the specific case of 
    the conduction band of a range of cubic materials (AlAs, AlP, GaN, GaP, SiC, ZnSe) with parameters found using ab-initio methods.\\ 
    The materials simulated had both isotropic and anisotropic conduction bands, a significant difference with respect to previously performed Diagrammatic Monte Carlo simulations, which were all 
    based on an isotropic model for the electron band.\\
    The results obtained are in agreement with previously computed values found in literature using different resolution methods and the obtained model 
    can be used as a starting point for more complex simulations.

    %\lipsum[1] % remove this line and write your own abstract
    
    \vfill
    \hspace{0pt}
\pagebreak