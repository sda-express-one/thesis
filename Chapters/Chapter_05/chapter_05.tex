\section{Methods and results}
The program developed was used to simulate the electron polaron of a range of cubic materials (zincblende and rocksalt structrures) both with 
isotropic (CBM at $\Gamma$) and anisotropic (CBM at $X$ point or along the $X-\Gamma$ high symmetry line) electronic band structure. For each one 
of them one single  phonon mode was considered since from the phonon band structure only one longitudinal optical mode is retrieved \cite{guster2021frohlich}.
\subsection{Input data}
\begin{table}[H]
    \centering
    \begin{tabular}{|c|c|c|c|c|c|c|c|c|}
        \hline
        \hline
        Material & $a$ (bohr) & edge & $m^*_\perp$ & $m^*_z$ & $\omega_{LO}$ (meV) & $\epsilon_\infty$ & $\epsilon_0$ & $\langle\alpha\rangle$\\
        \hline
        \hline
        AlAs-zb & $10.825$ & $X$ & $0.243$ & $0.897$ & $47.3$ & $9.49$ & $11.51$ & $0.184$ \\
        AlP-zb & $10.406$ & $X$ & $0.252$ & $0.809$ & $59.9$ & $8.12$ & $10.32$ & $0.184$ \\
        GaN-zb & $8.598$ & $\Gamma$ & $0.144$ & $0.144$ & $86.0$ & $6.13$ & $11.00$ & $0.345$ \\
        GaP-zb & $10.294$ & $X^*$ & $0.230$ & $1.062$ & $48.6$ & $10.50$ & $12.53$ & $0.152$ \\
        SiC-zb & $8.227$ & $X$ & $0.228$ & $0.677$ & $117.0$ & $6.97$ & $10.30$ & $0.280$ \\
        ZnSe-zb & $10.833$ & $\Gamma$ & $0.089$ & $0.089$ & $29.3$ & $7.35$ & $10.73$ & $0.276$ \\
        \hline        
    \end{tabular}
    \caption{Input values used for the simulations from \cite{miglio2020predominance} and \cite{guster2021frohlich}, the coupling strength are given as 
    averages over the solid angle $4\pi$.}
    \label{tab:inputs_params}
\end{table}
The 6 simulated materials were AlAs, AlP, GaN, GaP, SiC and ZnSe: the values of their effective masses, phonon energy, cell size, static dielectric tensor 
and optical dielectric tensor were computed with first-principle methods using GGA-PBE retrieved from \cite{miglio2020predominance} and \cite{guster2021frohlich} (Supplementary data). 
Their values are visible in \ref{tab:inputs_params} together with the computed average coupling strength as reference, which was found as 
\begin{equation}
    \langle\alpha\rangle=\left(\frac{1}{\epsilon_\infty}-\frac{1}{\epsilon_0}\right)^{-1}\sqrt{\frac{\langle m^*(\hat{\mathbf{q}})\rangle}{2\omega_{LO}}}.
\end{equation}
All the simulated electron polarons originated from an electron band which was isotropic (in $\Gamma$) or anisotropic uniaxial (in $X$ or in $X^*$, which is a short-hand notation for the high symmetry line
$X-\Gamma$), the longitudinal optical phonon energy were between the values of $30$meV and $117$meV covering an order of magnitude.\\
The lattice parameter (and thus the volume, computed as $a^3/4$) is reported for completeness even if it is by no means necessary in the simulation, in agreement 
with the large polaron approximation proper of the used Froehlich model.\\
The dielectric response $\epsilon^*$, which in the case of a single optical phonon mode simply reduces to 
\begin{equation}
    \epsilon^*=\left(\frac{1}{\epsilon_\infty}-\frac{1}{\epsilon_0}\right)^{-1},
\end{equation}
is independent of the nuclear masses since both $\epsilon_\infty$ (purely electronic) and $\epsilon_0$ (static) are. It measures the ionicity of the material and thus 
the strength of the electron-phonon coupling: for $\epsilon^*\to+\infty$ there is no coupling: this is the case for covalent materials such as diamond-C and diamond-Si where 
no polarization is possible between the atoms in the basis and $\epsilon_\infty=\epsilon_0$.
\subsection{Methods}
For each one of the studied materials 20 independent simulations were performed in order to get an estimation together with an error, 
the estimate was computed with a simple mean while the error with the formula for the sample variance:
\begin{equation}
    \bar{x}=\frac{1}{20}\sum_i x_i,\hspace{2cm}\Delta x=\sqrt{\frac{\sum_i\left(x_i-\bar{x}\right)^2}{19}}.
\end{equation}
The simulations were performed using a fixed $\tau$ value in order to have better-converged results for the ground state energy and the polaron 
effective masses. The $\tau$ value was set to 2000 for AlAS, AlP, SiC and ZnSe, 3000 for GaP, 10000 for GaN. As a rule of thumb, it is always better 
to have a greater fixed $\tau$ value for energy and effective mass computation, especially if the energy of the optical phonon is small or the coupling strength is high. This 
of course means that the computation times are inevitably longer.
For effective masses, in the isotropic case only the computed value for $m^*_{Px}$ were used for the result, in the anisotropic case only $m^*_{Px}$ ($m^*_\perp$) and $m^*_{Pz}$.
\subsection{Results}
\begin{table}[H]
    \centering
    \begin{tabular}{|c|c|c|c|}
        \hline
        \hline
        material & DMC (meV) & Froehlich (meV) & Feynman avg (meV)\\
        \hline
        \hline
        AlAs-zb & $-10.52\pm0.11$ & $-8.8$ & $-9.6$ \\
        AlP-zb & $-15.83\pm0.19$ &  $-14.0$ & $-15.1$ \\
        GaN-zb & $-30.11\pm0.42$ & $-29.6$ & $-29.7$ \\
        GaP-zb & $-8.59\pm0.24$ & -$7.4$ & $-8.3$ \\ 
        SiC-zb & $-34.72\pm0.42$ & $-32.7$ & $34.8$ \\
        ZnSe-zb & $-8.66\pm0.09$ & $-8.1$ & $-8.1$ \\
        \hline
    \end{tabular}
    \caption{Obtained polaron ground state energy with DMC ($0$ is the reference for the conduction band minimum), the values for Froehlich and Feynman were retrieved from \cite{guster2021frohlich}.}
    \label{tab:gs_obtained}
\end{table}
The results obtained for the ground state energy of the polaron (or Zero-Point Renormalization ZPR) are shown in \ref{tab:gs_obtained} together with the values computed with 
a generalized Froehlich formalism and the average (in the anisotropic case) values obtained with the Feynman variational approach, the values are to be interpreted as 
a renormalization of the conduction band minimum (and thus are all negative). The data obtained shows that all the computed parameters are estimated to be lower than 
what is found with the Feynman technique, as it should be since this is a variational technique which provides with an upper bound to the renormalization energy. In the case of an anisotropic electronic band, 
the Feynman result was found as 
\begin{equation}
    E^F_{Pavg}=\frac{2E^F_{P\perp}+E^F_{Pz}}{3}.
\end{equation}
\begin{table}[H]
    \centering
    \begin{tabular}{|c|c|c|c|c|c|c|}
        \hline
        \hline
         & \multicolumn{2}{|c|}{DMC} & \multicolumn{2}{|c|}{Froehlich} & \multicolumn{2}{|c|}{Feynman} \\
        \hline
        \hline
        material & $m^*_{P\perp}$ (a.u.) & $m^*_{Pz}$ (a.u.) & $m^*_{P\perp}$ (a.u.) & $m^*_{Pz}$ (a.u.) & $m^*_{P\perp}$ (a.u.) & $m^*_{Pz}$ (a.u.) \\
        \hline
        AlAs-zb & $0.2564$ & $0.9256$ & $0.2518$ & $0.9155$ & $0.2490$ & $0.9409$ \\
        AlP-zb & $0.2689$ & $0.8423$ & $0.2637$ & $0.8322$ & $0.2602$ & $0.8574$ \\
        GaN-zb & $0.1541$ & $0.1541$ & $0.1523$ & $0.1523$ & $0.1518$ & $0.1518$ \\
        GaP-zb & $0.2413$ & $1.0894$ & $0.2367$ & $1.0786$ & $0.2343$ & $1.1086$ \\
        SiC-zb & $0.2450$ & $0.7089$ & $0.2409$ & $0.7007$ & $0.2372$ & $0.7235$ \\
        ZnSe-zb & $0.0977$ & $0.0977$ & $0.0936$ & $0.0936$ & $0.09344$ & $0.0934$ \\
        \hline
    \end{tabular}
    \caption{Obtained polaron effective masses using DMC, the values for Froehlich and Feynman were retrieved from \cite{guster2021frohlich}.}
    \label{tab:eff_mass_polarons}
\end{table}
In \ref{tab:eff_mass_polarons} instead the polaron effective masses are shown (error estimations are not given since they are so small they can be neglected), as it can be seen 
in this case the situation is more varied and it is not clear which one of the computed values better estimates the actual polaron effective mass, 
also considering that for DMC we have employed the approximation \ref{effective_mass_approx} to obtain \ref{eff_mass_estimator_new}. Nevertheless the results obtained are still consistent 
with the symmetry of the model and with the values provided by the other approaches.




%BaO-rs & 10.566 & $X$ & 0.380 & 1.197 & 47.3 & 4.21 & 92.43 & 2.812 \\
%BN-zb & 6.746 & $X$ & 0.299 & 0.895 & 161.0 & 4.52 & 6.69 & 0.422 \\