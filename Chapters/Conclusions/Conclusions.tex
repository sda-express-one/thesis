\addcontentsline{toc}{section}{Conclusions}
\section*{Conclusions}
\markboth{Conclusions}{}
In this thesis we have seen how the Diagrammatic Monte Carlo approach can be used to simulate polaronic states inside real cubic 
materials and obtain estimates about the renormalization of the ground state energy of conduction band minima and about the polaron 
effective masses in a range of isotropic and anisotropic materials.\\ 
In Chapter 1 the failures of the Born-Oppenheimer approximation in modelling polaronic states were first presented, then the basic Froehlich model 
was derived and illustrated, together with the main approaches that are used to solve it. Then the generalized Froehlich model was illustrated, with a focus 
in particular on the non-degenerate anisotropic electron band case. Some analytical solution techniques were provided for this model.\\
In Chapter 2 the formalism provided by Quantum Many Body theory was used to obtain a Green's function (which is the solution to the Hamiltonian) which 
could be treated with Diagrammatic Monte Carlo, in particular by employing the Matsubara imaginary time formalism, which gets rid of the complex nature of the original 
Green's function and obtains a real non-negative representation, and using Wick's theorem, which allows us to treat the original interacting 
Green's function as a perturbative infinite series of non-interacting Green's functions and interaction vertices (both easy to evaluate).\\
In Chapter 3 the Monte Carlo method was explained illustrating the basic features of this numerical method, which depends on estimating 
expectation values of given quantities. Then the direct Monte Carlo method was briefly described, which directly models the investigated system, then the integration method, 
used to compute hard integrals, and the Markov chain Monte Carlo, a tool capable of generating random samples distributed as a target distribution after 
a given relaxation time using the Metropolis-Hastings algorithm. The Markov chain Monte Carlo method is at the basis of the Diagrammatic Monte Carlo approach, which 
extensively uses it to compute transitions from different diagrams using suitable updates.\\
In Chapter 4 the algorithm used for the Froehlich polaron simulation was discussed, from the way the free propagators and vertices were stored in the computer to the procedure used for 
each one of the implemented updates. The main estimators which can be used to retrieve information about the Green's function were also described.\\
In Chapter 5 the obtained results for a range of real materials were shown, together with the input data and basic information about the procedure which was undertaken.\\
The work done in this thesis can be used as a starting point for a more complete and complex simulation that takes into account degenerate electronic bands, electronic band anharmonicity and phonon mode 
anisotropy in order to simulate the electron and hole polarons in a wider range of materials.
