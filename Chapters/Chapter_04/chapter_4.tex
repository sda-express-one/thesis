\section{Diag MC for the Froehlich Polaron}
In chapter 2 we have seen that is possible to perturbatively expand the Matsubara Green's function for the Froehlich polaron 
in a series of integrals \ref{GF_expansion_integral}, while in chapter 3 we have shown that series of this type can be computed 
using a special type of Markov chain Monte Carlo called Diagrammatic Monte Carlo which is able to take into account all the possible 
perturbative expansions of the full function and sample from it.\\
We now focus on the specific Diagrammatic Monte Carlo technique employed for the Froehlich Polaron case. Let us consider the 
Froehlich Hamiltonian $H^{cFr}$ in the specific case of a single anisotropic electron band and multiple phonon modes. We take into account 
the non-interacting term
\begin{equation}
    H^{cFr}_0=\sum_{\mathbf{k}}\left(\frac{k^2}{2m(\hat{k})}-\mu\right)c^\dagger_{\mathbf{k}}c_{\mathbf{k}}+\sum_{\mathbf{q}j}\omega_{jLO}a^\dagger_{\mathbf{q}j}a_{\mathbf{q}j},
\end{equation}
where we are only taking into account the electron polaron (the only difference in the hole polaron case would be negative electron energy dispersion). The chemical potential 
$\mu$ is put inside the electron term as a renormalization constant (\textbf{fictitious potential renormalization} \cite{mishchenko2000diagrammatic})\cite{fehske2007computational} which can be tuned in order to obtain a non-divergent distribution that decays 
as an exponential for long $\tau$ in the stationary limit.\\\
The interaction term reads:
\begin{equation}
    V^{cFR}=\sum_{\mathbf{k},\mathbf{q}j}c^\dagger_{\mathbf{k+q}}c_{\mathbf{k}}\left(V^{cFr}_{\mathbf{q}j}a_q+V^{*cFr}_{\mathbf{q}j}a^\dagger_{-\mathbf{q}}\right),
\end{equation}
where the interaction vertex $V^{cFr}$ has formula
\begin{equation}
    V^{cFr}_{\mathbf{q}j}=\frac{i}{q}\frac{4\pi}{\Omega_0}\left(\frac{1}{2\omega_{jLO}V_{BvK}}\right)^{1/2}\frac{p_j}{\epsilon^{\infty}}.
\end{equation}
We need a technique to translate the Green's functions defined in chapter 2 to the formalism of Diagrammatic Monte Carlo. To this aim 
we identify the stationary distribution $Q(\left\{y\right\})$ with the two Green's function of interest: the one-electron Green's function 
$G(\mathbf{k},\tau)$, where no external phonon lines are present:
\begin{equation}
\begin{split}
    G(\mathbf{k},\tau)=\sum_{n=0,2,4,...}^{+\infty}\sum_{\xi_n}\int d\tau_1\cdots \int d\tau_n\int d\mathbf{q}_1 \frac{V_{q_1j_1}}{(2\pi)^3}\cdots\int d\mathbf{q}_{n/2}\frac{V_{q_{n/2}j_{n/2}}}{(2\pi)^3} \\
    \times D_n^{\xi_n}(\mathbf{k},\tau;\tau_1,...,\tau_{n},\mathbf{q}_1,...,\mathbf{q}_{n/2},j_1,...,j_{n/2}),
\end{split}
\end{equation}
and $P(\mathbf{k},\tau)$, a function that is the sum of the single electron Green's function and all the possible one-electron $N$-phonons Green's function 
configurations (weighted accordingly):
\begin{equation}
    P(\mathbf{k},\tau)=G(\mathbf{k},\tau)+\sum_{N=1}^{+\infty}G^{(N)}(\mathbf{k}\tilde{\mathbf{q}}_1,...,\tilde{\mathbf{q}}_N, j_1,...,j_N,\tau),
\end{equation}
where $G^{(N)}(\mathbf{k},\tau)$ is diagrammatically expanded in 
\begin{equation}
\begin{split}
   &G^{(N)}(\cdots)=\sum_{\xi_n}\int d\tau_1\cdots \int d\tau_n\int d\mathbf{q}_1\frac{V_{q_1j_1}}{(2\pi)^3}\cdots\int d\mathbf{q}_{n/2}\frac{V_{q_{n/2}j_{n/2}}}{(2\pi)^3}\\
    &\times D_n^{\xi_n}(\mathbf{k},\tilde{\mathbf{q}}_1,...,\tilde{\mathbf{q}}_N,\tilde{j}_1,...,\tilde{j}_N,\tau;\tau_1,...,\tau_n,\mathbf{q}_1,...,\mathbf{q}_{n/2},j_1,...,j_{n/2}).
\end{split}
\end{equation}
It is now important to distinguish between external variables ${y}$ and the internal ones $\{x_1,...,x_n\}$. We identify them as
\begin{equation}
\begin{split}
    \{&y\}\to \{\mathbf{k},\tilde{\mathbf{q}}_1,\tilde{j}_1,...,\tilde{\mathbf{q}}_N,\tilde{j}_N, \tau \},\\
    &n\to \{0,2,4,...\},\\
    \{&x_1,...,x_n\}\to \{\tau_1,...,\tau_n, \mathbf{q}_1, j_1,..., \mathbf{q}_{n/2}, j_{n/2}\}.
\end{split}
\end{equation}
The algorithm that computes our simulation must satisfy the ergodicity requirement of Markov chain Monte Carlo: for this reason it 
is necessary to implement updates that change the external variables $\{y\}$, the internal ones $\{x_i\}$, and the order of the diagram 
$n$.\\
We start by saying that no updates have been implemented in order to sample from the free electron momentum $\mathbf{k}$, which will be fixed 
for each individual simulation: this was made in order to have a more precise control about the region of $k-space$ that we want to simulate, in particular 
the $\mathbf{k}=0$ point where the electron is at the bottom (top) of the conduction (valence) band.\\
The chemical potential $\mu$ is also implemented in the simulations as a normalizing factor, in this way a free electron propagator 
becomes:
\begin{equation}
    G_0(\mathbf{k},\tau)\to G_0(\mathbf{k}, \mu, \tau)=e^{-(\epsilon(\mathbf{k})-\mu)\tau},
\end{equation}
the effect of which can be easily reversed by rescaling the result.\\
Due to the finite memory of computers, it is then necessary to fix for each simulation a maximum imaginary time value $\tau_{max}$, a maximum 
internal order $n_{max}$ (the number of internal phonon propagators would be $n_{max}/2$) and a maximum number of external phonon propagators $N_{max}$.\\
Having decided this, it is then necessary to state the electron effective mass on the three main axes $m^*_{x}$, $m^*_{y}$ and $m^*_{z}$ together with 
the number of phonon modes that couple to the electron band $n_{ph}$ together with their energy $\omega_{jLO}$ and their phonon mode polarities 
$p_j$ or another equivalent quantity from which their value can be recovered \cite{de2023high}.\\
Other quantities that needs to be specified are the size of the unit cell $\Omega_0$, the size of the Born-von Karman cell $V_{BvK}$, the optical 
dielectric tensor (a scalar in the case of cubic materials) $\epsilon_{\infty}$, the number of thermalization steps $N_{relax}$ set to achieve the 
stationary distribution condition and the number of simulation steps $N_{steps}$.\\
It is now important to define how the diagram was effectively modelled in the computer: to this aim three data structures were 
defined.
The \textbf{Vertex}: this data structure encapsulates all the data which describes the vertices of the electron-phonon interaction (also the diagram beginning and end)
together with all the relevant information about the associated phonon propagator, its components are:
\begin{itemize}
    \item $\tau_i$, the imaginary time value of the vertex, 0 and $\tau$ for the extrema
    \item \textbf{type}, integer parameter which differentiates between extrema (0), internal phonon vertices (+1 for outgoing phonon propagators 
    and -1 for incoming ones), external phonon vertices (+2 for outgoing phonon propagators and -2 for incoming ones).
    \item \textbf{linked}, integer value which shows to which other vertex the current vertex is linked to and thus the length of the 
    phonon propagator (whether we have an internal or external one).
    \item $\mathbf{q}_i$, the three momentum components of the free phonon propagator which is created or annihilated at the vertex.
    \item \textbf{index}, the phonon mode which is involved in the electron-phonon interaction at the vertex. 
\end{itemize}
The \textbf{Propagator}: this data structure represents the free electron propagators, it is defined by the three electron momentum components 
$\mathbf{k}_i$.\\
The \textbf{Effective\_Mass}: this data structure represents the effective mass $m^*_i(\hat{k})$ of the electron for each electron propagator $\mathbf{k}_i$, 
it is computed using the formula in \ref{effective_mass_anisotropic_relation}.\\
At the start of each simulation, an array of dimension $n_{max}+2N_{max}+2$ of Vertex data type is created, while an array of dimension 
$n_{max}+2N_{max}+1$ is created for the Propagator and Effective\_Mass data type. At the beginning of the simulation all the single elements of the three data types 
are initialized to default values with the exception of the first 2 Vertex elements and the first Propagator and Effective\_Mass 
element: the latter two are initialized using the $\mathbf{k}$ value given as input and by computing the related effective mass (default to 
1 in the case that $\mathbf{k}=0$), while the $\tau$ values of the first two Vertex elements are initialized to 0 and $\tau$ respectively, where 
$\tau$ is retrieved using the \textbf{Diagram Length} update, which will be explained in detail in the following section.
\subsection{Updates}
Updates are necessary to obtain a Markov chain for the system which is ergodic, there is no single way to implement them in order to 
achieve this aim and different choices have been proposed about the way new parameters are proposed and how updates are implemented 
\cite{mishchenko2000diagrammatic}\cite{hahn2018diagrammatic}, here it is thus shown just a proposal that is sufficient but is nevertheless not 
necessarily the best or fastest choice.\\
The first update that is described is the \textbf{Diagram Length} update (class I), which updates the length of the last electron propagator: it thus changes the +
value of the external variable $\tau$.\\
Since the last electron propagator is a free electron propagator, which decays exponentially with $\tau$ as
\begin{equation}
    G_0(\mathbf{k},\tilde{\mathbf{q}}_1,...,\tilde{\mathbf{q}}_N,\tau)=\exp{ \left\{-\left(\frac{p^2}{2m^*(\hat{p})}-\sum_{j}\omega_{jLO}n_{jph}-\mu \right)(\tau-\tau_{last})\right\}},
    \label{free_propagator_electron_final}
\end{equation}
with $\tau_{last}$ time of last vertex before the diagram end, and where we have:
\begin{equation}
\begin{split}
    \mathbf{p}=\mathbf{k}-\sum_{i=0}^N\tilde{\mathbf{q}}_i,\\
    \sum_j n_{jph}=N,
\end{split}
\end{equation}
with $N$ total number of external phonons in the current diagram and $j$ phonon mode index.\\
The weight ratio between two diagrams with length $\tau$ and $\tau'$ thus becomes:
\begin{equation}
    \frac{D_n(\cdots,\tau')}{D_n(\cdots,\tau)}=\frac{e^{-\Delta E (\tau'-\tau_{last})}}{e^{-\Delta E (\tau-\tau{last})}},
    \label{weights_diagram_length}
\end{equation}
where $\Delta E$ is the computed energy in \ref{free_propagator_electron_final}.\\
If we thus take an exponential distribution as the proposal distribution for new values of the imaginary time value $\tau'$ built as follows
\begin{equation}
    \tau'=\tau_{last}-\frac{\log{\left(1-r\right)}}{\Delta E},
\end{equation}
with $r$ uniform random variable in $[0,1]$ we obtain a new value that is always accepted apart from the case where $\tau'>\tau_{max}$. In fact the acceptance ratio is
\begin{equation}
    R_{\tau'\tau}=\frac{\exp{\left(\tau;\tau_{last},\Delta E\right)}e^{-\Delta E(\tau'-\tau_{last})}}{\exp{\left(\tau';\tau_{last},\Delta E\right)}e^{-\Delta E (\tau-\tau_{last})}}=\frac{\Delta E e^{-\Delta E(\tau-\tau_{last})}e^{-\Delta E (\tau'-\tau_{last})}}{\Delta E e^{-\Delta E(\tau'-\tau_{last})}e^{-\Delta E (\tau-\tau_{last})}}=1,
\end{equation}
which is always accepted.\\
The next update to be described is the \textbf{add internal phonon}, a class II update. The new (internal) variables that have to be proposed 
are the time value of the outgoing vertex $\tau'$, the time value of the incoming vertex $\tau''$, the phonon mode index $j$ and the phonon propagator 
momentum $\mathbf{q}$. If proposing the diagram means that $n+2>n_{max}$ maximum allowed internal order, then the update is rejected.\\
An electron propagator (from the first one to the last for a total of $n+1$ possible choices) is chosen at random, then the value of $\tau'$ is generated 
using a uniform distribution between $\tau_{left}$ and $\tau_{right}$ vertices of the chosen free electron propagator.\\
The phonon mode $j$ is then chosen at random between the ones given in input, and the value of the second vertex of the phonon propagator $\tau''$ is generated using the
following exponential distribution
\begin{equation}
    P(\tau''|\tau',j)=\tau'-\frac{\log(1-r)}{\omega_{jLO}},
\end{equation}
where $\tau''$ is not restrained to any particular free electron propagator. The update is straight-up rejected if $\tau''>\tau$ length of the 
full diagram or if it is too close to another vertex ($|\tau''-\tau_i|<10^{-9}$).\\
The phonon momentum values $\mathbf{q}$ are then proposed using a probability $P(\mathbf{q}|\tau',\tau'',j)$ dependent on a gaussian distribution with 
mean $0$ and variance $(\tau''-\tau')^{-1}$.\\
The ratio of the two diagrams are
\begin{equation}
\begin{split}
    &\frac{D_{(n+2)}(\mathbf{k},\tau,...,\tau',\tau'',\mathbf{q},...)}{D_{n}(\mathbf{k},\tau,...)}=\\
    =&\exp{\left\{-\left(\sum_i\epsilon(\mathbf{k}_i-\mathbf{q})-\epsilon(\mathbf{k}_i)\Delta\tau_i-\omega_{LO}(\tau''-\tau')\right)\right\}}|V_{\mathbf{q}j}|^2d\tau'd\tau''\frac{d\mathbf{q}}{(2\pi)^3},
\end{split}
\end{equation}
where the sum over $i$ is extended over all the phonon propagators between the two phonon vertices $\tau'$ and $\tau''$, while $\Delta\tau_i$ is computed as $\tau_i-\tau_{i-1}$ where the two 
extrema are $\tau'$ and $\tau''$. The infinitesimal are required because they are not cancelled in the ratio since they are new proposed variables.\\
The full distribution from which the new values are sampled is written as
\begin{equation}
\begin{split}
    &P(\tau',\tau'',\mathbf{q})=P(\tau')P(\tau''|\tau',j)P(\mathbf{q}|\tau',\tau'',j)=\\
    &=\underbracket[1.5pt][3pt]{\frac{1}{\tau_{right}-\tau_{left}}}_{P(\tau')}\underbracket[1.5pt][3pt]{\omega_{jLO}e^{-\omega_{jLO}(\tau''-\tau')}}_{P(\tau''|\tau',j)}
    \underbracket[1.5pt][3pt]{\left(\frac{\tau''-\tau'}{2\pi}\right)^{3/2}e^{-\frac{q^2}{2}(\tau''-\tau')}}_{P(\mathbf{q}|\tau',\tau'',j)}.
\end{split}
\end{equation}
The acceptance ratio thus becomes:
\begin{equation}
    R_{add}=\frac{p_A}{p_B}\frac{D_{n+2}(\mathbf{k},\tau,...,\tau',\tau'',\mathbf{q},...)}{D_n(\mathbf{k},\tau,...)P(\tau',\tau'',\mathbf{q})},
\end{equation}
the context factors $p_A$ and $p_B$ take into account the number of free electron propagators from which a vertex can be generated and the total number of internal phonon propagators 
which can be removed, their ratio is
\begin{equation}
    \frac{p_A}{p_B}=\frac{p_{(rem\hspace{2pt}int)}}{p_{(add\hspace{2pt}int)}}\frac{n+2N+1}{n/2+1},
\end{equation}
for example at order $n=2$ and 1 external phonon we can choose from $2+2\cdot1+1=5$ electron propagators, while to go back to 
order 2 from order $n+2=4$ we can choose between $2/2+1=2$ phonon propagators. The two variables $p_{(rem\hspace{2pt}int)}$ and 
$p_{(add\hspace{2pt}int)}$ measure the probability of choosing the add internal and remove internal updates in the main Monte Carlo simulation.\\
The \textbf{remove internal phonon} is also a class II update, no new parameters are proposed but instead a random phonon propagator with 
vertices at $\tau'$ and $\tau''$ and momentum $\mathbf{q}$ is chosen to be removed. The update is automatically rejected if the internal order $n=0$ and 
is accepted with acceptance ratio $R_{rem}=1/R_{add}$ calculated for the add internal phonon update.\\
\subsection{Collected quantities and exact estimators}