\addcontentsline{toc}{section}{Introduction}
\section*{Introduction}
\markboth{Introduction}{}
In Condensed Matter Physics, polarons are coherent states that are formed when an electron (or a hole) couples to a phonon bath in an ionic material (specifically the optical phonons): 
this usually happens at the bottom of the conduction band for the electron polaron and at the top of the valence band for hole polarons and leads to a renormalization of the electron 
ground state energy and effective mass.\\
This coupling cannot be described using standard ab-initio methods such as DFT, which decouples the ionic degrees of freedom from the electronic ones using 
the Born-Oppenheimer approximation and uses the former to provide an energy landscape for the electrons, and thus the formalism provided by Quantum Many Body theory 
is necessary to understand the physics underlying the phenomenon under study.\\
Two different models for the polaron exist: the Froehlich polaron \cite{frohlich1954electrons}, which assumes that the characteristic size $d$ of the electron with its phonon cloud, 
which together form the polaron, is much greater than the lattice spacing $a$ of the material, and the Holstein polaron \cite{holstein1959studies}\cite{holstein1959studies01}, which is suitable to describe 
polarons where the actual crystalline arrangement of the material cannot be neglected (and thus $d$ is of the order of the lattice parameter).\\
In this thesis we will be focusing on the first case, and we will provide a way to compute relevant quantities about Froehlich polarons in a range 
of real materials using the Diagrammatic Monte Carlo method. For this scope, a model which includes anisotropies in the electron band and multiple phonon modes 
will be provided.\\
Here the organization of the thesis is briefly explained:
\begin{itemize}
    \item In Chapter 1 the standard Froehlich model is presented and it is explained how its Hamiltonian is derived together with the main techniques to 
    solve it. It is then explained how this relatively simple model can be generalized to have multiple anisotropic electronic bands 
    and multiple optical phonon modes without altering the basic features of the model from which it was derived.
    \item In Chapter 2 the main Many-Body techniques employed to make the Froehlich model treatable using the Diagrammatic Monte Carlo 
    method are explained: this includes using Green's functions as solutions for the Froehlich Hamiltonian together with the Matsubara imaginary time 
    formalism and Wick's theorem to perturbatively expand the interacting Green's function in an integral series of non-interacting terms. In this way it becomes 
    possible to use Feynman diagrams to solve the model, with the added bonus of obtaining a Green's function that is real and non-negative (thanks to the imaginary time 
    formalism).
    \item In Chapter 3 the Monte Carlo method is explained, from Monte Carlo integration to Markov chain Monte Carlo, a technique which can 
    be used to obtain random samples distributed as a target distribution, even if it unnormalized, using the Metropolis-Hastings algorithm. Markov chain Monte Carlo is at the basis 
    of the Diagrammatic Monte Carlo method, which also needs to implement ways to add or remove internal variables (since we are describing a system 
    which can be described as an expansion of terms each with a different weight).
    \item  In Chapter 4 the implementation of the algorithm is discussed: the basic features of the DMC simulation, the way free electron propagators 
    and free phonon propagators were modelled in the computer, the updates that were implemented in order to obtain an ergodic simulation, and the main estimators 
    that were used in order to obtain the results, among these the most important are the exact ground state energy estimator and the exact 
    effective mass estimator.
    \item In Chapter 5 the results obtained together with the input parameters employed are compared to the same quantities computed with different methods (namely 
    the perturbative method for generalized cubic Froehlich and the Feynman variational approach).
\end{itemize}